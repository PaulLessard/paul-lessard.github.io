% Site wide macro definitions

% *****************************************************************
% ** Description: Various macros for box twiddling
% *****************************************************************

\makeatletter

\def\clap#1{\hbox to 0pt{\hss#1\hss}}
\def\mathllap{\mathpalette\mathllap@}
\def\mathrlap{\mathpalette\mathrlap@}
\def\mathclap{\mathpalette\mathclap@}
\def\mathllap@#1#2{\llap{$\m@th#1{#2}$}}
\def\mathrlap@#1#2{\rlap{$\m@th#1{#2}$}}
\def\mathclap@#1#2{\clap{$\m@th#1{#2}$}}

\def\makeslashed#1#2#3#4#5{#1{\mathpalette{\sla@{#2}{#3}{#4}}{#5}}}

\def\@mathlower#1#2#3{\setbox0=\hbox{$\m@th#2{#3}$}\lower#1\ht0\box0}
\def\mathlower#1#2{\mathpalette{\@mathlower{#1}}{#2}}

\def\@mathscale#1#2#3{\scalebox{#1}{$\m@th#2{#3}$}}
\def\mathscale#1#2{\mathpalette{\@mathscale{#1}}{#2}}

\def\@mathadj#1#2#3#4{\setbox0=\hbox{\scalebox{#1}{$\m@th#3{#4}$}}\lower#2\ht0\box0}
\def\mathadj#1#2#3{\mathpalette{\@mathadj{#1}{#2}}{#3}}

\makeatother

% ***************************************************************
% ** Description:      Some useful mathematical operators.
% ***************************************************************

% ordinal stuff

%\newcommand{\st}{^{\text{st}}}
%\newcommand{\nd}{^{\text{nd}}}
%\newcommand{\rd}{^{\text{rd}}}
%\renewcommand{\th}{^{\text{th}}}

% General mathematical connectives etc.

\newcommand{\defeq}{\coloneq}

% Operators

\makeatletter
\def\newmop{\@ifstar{\@newmop m}{\@newmop o}}
\def\@newmop#1{\@ifnextchar[{\@@newmop #1}{\@@@newmop #1}}
\def\@@newmop#1[#2]{\@declmathop #1#2}
\def\@@@newmop#1#2{\expandafter\@declmathop\expandafter #1\csname #2\endcsname{#2}}
\makeatother

% General operations on maps etc.
\newmop{dom}
\newmop{cod}
\newmop{id}
\newmop{obj}
\newmop{arr}
\newmop{ev}
\newmop{el}
\newmop{im}

% ***************************************************************
% ** Description:      General categorical notations
% ***************************************************************

\newcommand{\comma}{\mathbin{\downarrow}}

%iso-commas

%\newcommand{\itimes}{%
%  \mathrel{\vbox{\offinterlineskip\ialign{%
 %   \hfil##\hfil\cr
 %   $\scriptscriptstyle\sim$\cr

%    $\times$\cr
%}}}}
\newcommand{\icomma}[1]{\mathbin{\mathop{\dottimes}\limits_{#1}}}

\newcommand{\timesdot}{%
            \mathrel{\raisebox{-.1em}{%
            \reflectbox{\rotatebox[origin=c]{180}{$\dottimes$}}}}}

\newcommand{\unit}{\eta}
\newcommand{\counit}{\epsilon}

\newcommand{\pwr}{\pitchfork}

%tensor and cotensor bifunctors
\newcommand{\ctns}[2]{{{#2}^{#1}}}%weight first, then variable
\newcommand{\tns}[2]{{#1\otimes #2}}

%weighted limit and colimit bifunctors
\newcommand{\wlim}[3][]{{\lim\nolimits_{#2}^{#1}{#3}}}
\newcommand{\hlim}[2]{{\lim\nolimits_{#1}^{\simeq}{#2}}}
\newcommand{\wcolim}[3][]{{\colim\nolimits_{#2}^{#1}{#3}}}
\newcommand{\leibwlim}[3][]{{\widehat{\lim}_{#2}^{#1}{#3}}}
\newcommand{\leibwcolim}[3][]{{\widehat{\colim}_{#2}^{#1}{#3}}}
\newmop*{oplax}

% for yoneda - this is the Unicode character "HIRAGANA LETTER YO" (よねだ).
\newcommand{\yo}{\text{\japanese よ}}

% Duals / Superscripted postfix ops

\newcommand{\op}{^{\mathord{\text{\rm op}}}}
\newcommand{\co}{^{\mathord{\text{\rm co}}}}
\newcommand{\coop}{^{\mathord{\text{\rm coop}}}}

%for fiber and cofiber sequences
%\newmop{ker}
\newmop[\fiber]{fib}
\newmop{coker}
\newmop{cofib}

% ***************************************************************
% ** Description:      Standard categories.
% ***************************************************************

\newcommand{\catfour}{{\Bbbfour}}%texdoc unimath-symbols.pdf
\newcommand{\catthree}{{\Bbbthree}}%texdoc unimath-symbols.pdf
\newcommand{\cattwo}{{\Bbbtwo}}
\newcommand{\catone}{{\Bbbone}}
\newcommand{\catn}{{\mathbb{n}}}
\newcommand{\catnone}{{\catn\!+\!\catone}}
\newcommand{\catntwo}{{\catn\!+\!\cattwo}}
\newcommand{\iso}{{\BbbI}}

\newcommand{\Del}{\mbfDelta}
\newcommand{\eDel}{\mbfDelta_{\mathord{\text{\rm epi}}}}
\newcommand{\Delt}{\mbfDelta_{\top}}%formerly_{\infty}
\newcommand{\Delb}{\mbfDelta_{\bot}}%formerly_{-\infty}
\newcommand{\Delbt}{\mbfDelta_{\bot,\top}}%for the intersection

\newcommand{\NN}{{\mathbb{N}}}%for the graph underlying \mbfomega

\newcommand{\Thet}{\mbfTheta}
% *****************************************************************
% ** Description: Notation for quasi-categories and cosmoi
% *****************************************************************

\newcommand{\qop}[1]{{\mathord{\mathsf{#1}}}}

%enriched (i.e. all large) categories.
\newcommand{\ec}[1]{{\mathord{\mathcal{#1}}}}
\let\eop=\ec
\newcommand{\cl}[1]{{\mathord{\symtt{#1}}}}

%cosmoi
\newcommand{\qCat}{\eop{QCat}}
\newcommand{\Sp}[1]{{\Thet_{#1}\text{-}\eop{Sp}}}
\newcommand{\Segal}{\eop{Segal}}
\newcommand{\Rezk}{\eop{Rezk}}%for Rezk objects
\newcommand{\CSS}{\eop{CSS}}%###also consider calling Rezk?
\newcommand{\Cat}{\eop{Cat}}%###the cosmos version, which we might never use
\newcommand{\Comp}[1]{#1\text{-}\eop{Comp}}%saturated {#1}-trivial weak complicial sets
\newcommand{\coCart}{\eop{coCart}}
\newcommand{\Cart}{\eop{Cart}}
\newcommand{\Fib}{\eop{Fib}}
\newcommand{\vDblCat}{\eop{vDblCat}}
\newcommand{\VE}{\eop{VE}}%for folds signature for virtual equipments

\newcommand{\dCart}{\eop{dCart}}%discrete cartesian fibrations
\newcommand{\dCoCart}{\eop{dCoCart}}%discrete cocartesian fibrations

\newcommand{\Lari}{\eop{Lari}}
\newcommand{\Rari}{\eop{Rari}}
\newcommand{\Stab}{\eop{Stab}}

%discrete cosmoi
\newcommand{\Disc}{\eop{Disc}}
\newcommand{\Kan}{\eop{Kan}}
\newcommand{\Mod}{\eop{Mod}}

%related categories
\newcommand{\Space}{\eop{Space}}
\newcommand{\sSet}{\eop{sSet}}
\newcommand{\cSet}{\eop{cSet}}%cubical sets; for a note about unstraightening
%\newcommand{\ssSet}{\eop{SSSet}}
\newcommand{\msSet}{\sSet^+}%for MARKED simplicial sets; replacing STRAT.
\newcommand{\asSet}{\sSet_+}%for AUGMENTED simplicial sets
\newcommand{\mzsSet}{\sSet^{+_0}}%for MARKED simplicial sets with marked vertices

%categories
\newcommand{\Set}{\eop{Set}}
\newcommand{\Gph}{\eop{Gph}}
\newcommand{\sCat}{\sSet\text{-}\Cat}
\newcommand{\msCat}{\msSet\text{-}\Cat}
\newcommand{\mzsCat}{\mzsSet\text{-}\Cat}
\newcommand{\sCptd}{\sSet\text{-}\eop{Cptd}}
\newcommand{\msCptd}{\msSet\text{-}\eop{Cptd}}
\newcommand{\sRCptd}{\sSet\text{-}\eop{rCptd}}
\newcommand{\kanCat}{\Kan\text{-}\Cat}%for the subcategory of kan-complex enriched simplicial cats?
\newcommand{\twoCat}{2\text{-}\Cat}
\newcommand{\eCat}[1]{{#1}\text{-}\Cat}%for enriched categories
\newcommand{\FrExt}{\eop{FrExt}}

%enriched things/Kan-/quasi-###the "c" standing for "cosmos"?
\newcommand{\cA}{\ec{A}}
\newcommand{\cB}{\ec{B}}
\newcommand{\cC}{\ec{C}}
\newcommand{\cD}{\ec{D}}
\newcommand{\cE}{\ec{E}}
\newcommand{\cF}{\ec{F}}
\newcommand{\cG}{\ec{G}}
\newcommand{\cH}{\ec{H}}
\newcommand{\cI}{\ec{I}}%for inverse categories
\newcommand{\cJ}{\ec{J}}%for indexing shapes that might be enriched, I guess? or maybe use \cA?
\newcommand{\cK}{\ec{K}}
\newcommand{\cL}{\ec{L}}
\newcommand{\cM}{\ec{M}}
\newcommand{\cN}{\ec{N}}
\newcommand{\cP}{\ec{P}}
\newcommand{\cQ}{\ec{Q}}
\newcommand{\cR}{\ec{R}}%symbol seems to be incorrect in alternate 2 of Asana.
\newcommand{\cS}{\ec{S}}%for Kan enriched
\newcommand{\cT}{\ec{T}}
\newcommand{\cU}{\ec{U}}
\newcommand{\cV}{\ec{V}}%for the base of enrichment (=cartesian closed since why not?)
\newcommand{\cW}{\ec{W}}%for the other base of enrichment (=cartesian closed

%###trial notation for the homotopy 2-category
\newcommand{\h}{\mathfrak{h}}

%### virtual equipment of modules
\newcommand{\MMod}{\mathord{\mathbb{M}\mathrm{od}}}
\newcommand{\CC}{{\mathbb{C}}}% generic virtual equipments
\newcommand{\PP}{{\mathbb{P}}}

% ###classes in a model category/cat of fibrant objects in case we want to change the font
\newcommand{\CF}{\cl{C}}
\newcommand{\FB}{\cl{F}}
\newcommand{\WE}{\cl{W}}
\newcommand{\classA}{\cl{A}}
\newcommand{\classB}{\cl{B}}
\newcommand{\classC}{\cl{C}}
\newcommand{\classE}{\cl{E}}
\newcommand{\classF}{\cl{F}}
\newcommand{\classI}{\cl{I}}
\newcommand{\classJ}{\cl{J}}
\newcommand{\classK}{\cl{K}}
\newcommand{\classL}{\cl{L}}
\newcommand{\classM}{\cl{M}}
\newcommand{\classR}{\cl{R}}

% Various hom notations.
\newcommand{\Fun}{\qop{Fun}}%mapping spaces in cosmoi
\newcommand{\hFun}{\qop{hFun}}%hom categories in homotopy 2-categories
\newcommand{\cFun}[1]{{\qop{Fun}_{#1}^{\cart}}}%mapping spaces in cartesian cosmoi
\newcommand{\Hom}{\qop{Hom}}%all other homs in large/enriched categories
\newcommand{\Map}{\qop{Map}}%mapping space (module) for ∞-categories
\newcommand{\Icon}{\qop{Icon}}%hom simplicial sets between simplicial categories

% ***************************************************************
% ** Description:      Abstract homotopy theory.
% ***************************************************************

% Elementary operators in the theory of (stratified) simplicial sets.

\newcommand{\face}{\delta}
\newcommand{\degen}{\sigma}

% face by vertices (fbv)
\newcommand{\sembl}{\mathopen{\mathord[\mkern-3mu\mathord[}}
\newcommand{\sembr}{\mathclose{\mathord]\mkern-3mu\mathord]}}
\newcommand{\fbv}[1]{\{{#1}\}}

% other simplicial notation

\newmop{sk}
\newmop{cosk}
\newmop{res}
\newmop{trunc}
\newmop{ir}%for interval representation

\newcommand{\join}{\mathbin\star}
\newcommand{\fatjoin}{\mathbin\diamond}
\newmop{dec}
\newmop{fatdec}
\newmop{slc}
\newmop{fatslc}
\newmop{cyl}
\newcommand{\slicel}[2]{\vphantom{#2}^{{#1}/}\mkern-2mu{#2}}
\newcommand{\slicer}[2]{{#1\mkern-1mu}_{{}/{#2}}}

%for the lax slice
%\newcommand{\sslice}{{\mathord{\mathord{/}\!\mathord{/}}}}
\newcommand{\laxslicel}[2]{\vphantom{#2}^{{#1}\sslash}\mkern-2mu{#2}}
\newcommand{\laxslicer}[2]{{#1\mkern-1mu}_{{}{\sslash}{#2}}}

% Boundary operator.
\newcommand{\bound}{\partial}

% Simplicial dual.
\newcommand{\sdual}{^\circ}

%homotopy categories

%\newmop{ho} %###tweak: too much space now
\newcommand{\ho}{\qop{h}}

% Leibniz...
% ... for binary operators like join and product
\newcommand{\leib}[1]{\mathbin{\widehat{#1}}}
% ... for prefix operators like lim, colim and decalage.
\newcommand{\uleib}[1]{\widehat{#1}}

% probably should change the first of those to a normal hat and grep the
% text to make sure that the right variant is being used in various places.

% nerves etc

\newcommand{\hC}{\mfrakC}
\newcommand{\hN}{\mfrakN}
\newcommand{\mhC}{\mfrakC^+\!}
\newcommand{\mhN}{\mfrakN^+}
